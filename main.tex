\documentclass[12pt]{article} 
\usepackage[utf8]{inputenc}
\usepackage[spanish]{babel}
\usepackage{graphicx} 
\usepackage{geometry} 
\usepackage{float} 
\usepackage{amsmath}
\usepackage{hyperref}
\usepackage{float}

\geometry{letterpaper, margin = 2cm}   

\title{Asistente de investigación - MCIC}

\begin{document}

\begin{center}
{\Large\bfseries Asistente de investigación - MCIC} \\
\end{center}

\begin{center}
\textbf{Diego Andrés Garzón Girón} \\
\href{mailto:dagarzong@udistrital.edu.co}{dagarzong@udistrital.edu.co} \\
Cel/WhatsApp: \textit{3058518131}
\end{center}

\section{Contexto de la propuesta}
\subsection{Resumen}
En cuanto al resumen que se presenta quedo totalemnte interesado en el proyecto ya que a lo largo de mi carrera siempre me ha gustado el campo de la programación y en especial, en este semestre académico he tenido la posibilidad de cursar la materia de Geoestadística y se me ha hecho una asignatura indispensable para el ámbito y formación de un Ingeniero Catastral y Geodesta, me parece que las técnicas y los saberes que he aprendido allí son totalmente aplicables a esta investigación. Al hacer mención de que se trabajará con datos georreferenciados y con diferentes técnicas de aprendizaje me es muy grato saber que podría estar aplicando todo lo que he aprendido en la academia en un proyecto de investigación en la vida real enfocándolo a diferentes tipos de productos o fenómenos como se menciona en el documento compartido. 
\subsection{Objetivos del proyecto}
Me siento en toda la capacidad y disposición de apoyar en los cuatro objetivos del proyecto mencionados en el documento, y si con el pasar del tiempo y de la investigación llegasen a ser más estaré en toda la voluntad de abordarlos.

\section{Requisitos y prueba de selección}
\subsection{Requisitos}
Haré unos breves comentarios por cada uno de los ítems que en esta parte se mencionan:
\begin{itemize}
    \item \textbf{R, Python y SQL: }De todos esos lenguajes tengo idea, Python lo he utilizado desde que vi programación básica y ha sido el lenguaje de programación con el cuál más he trabajado, en cuanto a SQL lo utilicé bastante en las asignaturas de Bases de Datos y Bases de Datos Espaciales, además de un curso que hice hace unos años de desarrollador web, en cuanto a R es un lenguaje que lo conozco desde hace bastante tiempo, pero ha sido hasta este año en donde he experimentado más y lo he trabajado de manera más recurrente.
    \item \textbf{Estadística, Ciencia de Datos: }Son dos ramas que según mi forma de ver el mundo indicarán tanto el futuro de nuestra carrera como el del mundo en general, de esa misma forma trato, sobre todo para este tiempo de tomarme estos campos del conocimiento con total seriedad e involucrarme cada vez más en ellos, con cursos online, libros, o incluso inscribiendo materias en pro de mejorar mis conocimientos aquí como lo son Bases de Datos Espaciales y Geoestadística.
    \item \textbf{LateX: }LateX es una herramienta poderosa que utilizo hace relativamente poco, pero la he ido adaptando a mis necesidades y he orientado mi formación también en aprender habilidades como esta que siento que permiten entregar productos de mejor calidad.    
\end{itemize}

\subsection{Condiciones}
Al igual que en la sección anterior daré mi punto de vista de uan manera corta y concisa de cada una de las condiciones expuestas en el documento compartido en el orden que allí se encuentran.

\begin{itemize}
    \item El tema del trabajo de grado es algo que personalmente me motiva demasiado, ya que estoy en proceso de finalizar la carrera y por ende ya también estoy pensando en mi trabajo de grado, tenía una idea, que finalmente no se materializó con el docente Carlos Castro, que es una persona con la cual considero que llevo buena relación y estoy inscrito en su semillero de manera oficial hace unos nueve meses ya.
    \item Tengo total disposición para las reuniones remotas, este semestre mi carga académica es de solo tres materias, por lo cual creo que podría también dedicar más tiempo al proyecto.
    \item Me considero como alguien responsable y afín a la investigación, es ese el motivo por el cuál también me presento a esta convocatoria y la razón por la cuál soy participante activo de dos semilleros de investigación en la Universidad.
    \item El tener la posibilidad de aprender es otro de los puntos claves que más me gustan de este proyecto, por ende estoy muy dispuesto a aprender.
    \item Para los tres últimos ítems solo mencionaré que estoy de acuerdo y agradezco la forma en que se reconocen las actividades del ayudante de investigación, y por supuesto me acogería a ellas en caso dado de quedar seleccionado.
\end{itemize}

\section{Prueba técnica}
\subsection{Análisis e interpretación de datos}

Con los datos que se muestran se puede evidenciar que:  

\begin{itemize}
    \item Para “Murder”, los valores oscilan entre 0.8 y 17.4, con una mediana de 7.25 y una media de 7.79, lo que la hace la variable con el menor rango de valores.
    \item Para “Assault”, los valores oscilan entre 45 y 337, con una mediana de 159 y una media de 170.8, siendo la que tiene mayor variabilidad y mayor rango de valores.
    \item Para “Rape”, los valores oscilan entre 7.3 y 46, con una mediana de 20.1 y una media de 21.23.
\end{itemize}

En cuanto a las anomalías se puede comparar el tamaño de las diferentes barras, observando que hay algunas que son pronunciadamente más altas que las demás. Esto podría indicar que hay valores atípicos en los datos. Por ejemplo, en la variable “Murder”, hay una barra que llega hasta 32 en el eje y, lo cual es bastante más alto que las otras barras y es de lejos la barra con mayor alcance en este eje.
Para hablar de las tendencias podemos ver el gráfico de densidad, aquí se observa el comportamiento de las tres variables, “Murder” y “Assault” parecen tener distribuciones sesgadas a la derecha, ya que sus picos están a la izquierda y la cola se extiende hacia la derecha. Por otro lado, “Rape” parece tener una distribución más simétrica, ya que su pico está más centrado.
Si nos basamos en estos hallazgos, parece que “Assault” es la variable con mayor variabilidad y posiblemente también con mayor cantidad de valores atípicos. “Murder” y “Assault” parecen tener distribuciones sesgadas a la derecha, lo que indica que la mayoría de los valores son bajos y hay algunos valores altos. Por otro lado, “Rape” parece tener una distribución más simétrica, lo que indica que los valores están más equilibrados.

\subsection{Manejo censo de edificaciones}
Aunque el código irá adjunto al correo que se envíe con este archivo me permitiré mencionar algunas cosas del ejercicio por este medio, sobre todo en materia de los gráficos generados, ya que el código quedó ampliamente comentado:

El primer gráfico que se obtuvo fue el de la Figura \ref{fig:4-nulos} Y representa a priori cuales eran los resultados que salían al mostrar exclusivamente lo que para el programa eran las 4 variables con más datos vacíos

\begin{figure}[H] %Nos permite que la imagen se coloque en la posición que tiene en el documento
    \centering %centrado imagen
    \includegraphics[width = 10cm]{"Grafico 1 4 variables con mas datos nulos.jpeg"}
    \caption{Cuatro variables con más datos faltantes}
    \label{fig:4-nulos}
\end{figure}
Luego se miró de una manera general como se comportaban estos datos faltantes y se puede evidenciar en la Figura \ref{fig:TotalVariables}
\begin{figure}[H] %Nos permite que la imagen se coloque en la posición que tiene en el documento
    \centering %centrado imagen
    \includegraphics[width = 10cm]{Grafico 2 total variables con mas datos nulos.jpeg}
    \caption{Variables con más datos faltantes}
    \label{fig:TotalVariables}
\end{figure}

Luego procedió a eliminar esas tres variables que tenían datos extremadamente altos comparados con las otras columnas, y luego, como lo muestra la Figura \ref{fig:Grafico-3} todas las demás columnas tenían 20 datos faltantes a excepción del Año de Censo

\begin{figure}[H] %Nos permite que la imagen se coloque en la posición que tiene en el documento
    \centering %centrado imagen
    \includegraphics[width = 10cm]{grafico3.jpeg}
    \caption{Gráfico sin las tres variables con más datos faltantes}
    \label{fig:Grafico-3}
\end{figure}
El último gráfico representa el conteo de las edificaciones por año utilizando la variable NROEDIFIC que según el diccionario de datos suministrado corresponde a  Número de edificaciones en construcción corresponde al número de estructuras que se están  construyendo en el momento del operativo. 
\begin{figure}[H] %Nos permite que la imagen se coloque en la posición que tiene en el documento
    \centering %centrado imagen
    \includegraphics[width = 10cm]{grafico4.jpeg}
    \caption{Conteo de las edificaciones por año}
    \label{fig:Grafico-4}
\end{figure}

Sin más por agregar agradezco la atención brindada y cualquier comentario de retroalimentación a mis redes de contacto especificadas al inicio de este documento, le deseo un excelente día e inicio de semana y espero podamos llegar a trabajar juntos.


\end{document}